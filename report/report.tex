% TEMPLATE for Usenix papers, specifically to meet requirements of
%  USENIX '05
% originally a template for producing IEEE-format articles using LaTeX.
%   written by Matthew Ward, CS Department, Worcester Polytechnic Institute.
% adapted by David Beazley for his excellent SWIG paper in Proceedings,
%   Tcl 96
% turned into a smartass generic template by De Clarke, with thanks to
%   both the above pioneers
% use at your own risk.  Complaints to /dev/null.
% make it two column with no page numbering, default is 10 point

% Munged by Fred Douglis <douglis@research.att.com> 10/97 to separate
% the .sty file from the LaTeX source template, so that people can
% more easily include the .sty file into an existing document.  Also
% changed to more closely follow the style guidelines as represented
% by the Word sample file. 

% Note that since 2010, USENIX does not require endnotes. If you want
% foot of page notes, don't include the endnotes package in the 
% usepackage command, below.



% Good read: https://www.usenix.org/legacy/events/samples/template.la
\documentclass[letterpaper,twocolumn,10pt]{article}
\usepackage{usenix,epsfig,endnotes}
\begin{document}

%don't want date printed
\date{}

%make title bold and 14 pt font (Latex default is non-bold, 16 pt)
\title{ RL-Competition: Lambda sarsa agent\\ \small a submission to the competition in ``Decision making under uncertainty``}

\author{
{\rm John Karlsson}\\
kajohn@student.chalmers.se
\and
{\rm Oscar Carlsson}\\
coscar@student.chalmers.se
\and
{\rm Oskar Lindgren}\\
oskar.lindgren@chalmers.se
}

\maketitle

% Use the following at camera-ready time to suppress page numbers.
% Comment it out when you first submit the paper for review.
\thispagestyle{empty}


\subsection*{Abstract}
\input import/abstract.tex

\section{Introduction}
\input import/intro.tex

\section{Environments}
\input import/environments.tex

\section{The Agent}
\input import/agent.tex

\section{Results}
\input import/results.tex


{\footnotesize \bibliographystyle{acm}
\bibliography{sample}}


\theendnotes

\end{document}







