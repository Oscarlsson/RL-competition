\paragraph{Complementing the core mechanism}
In some environments, the state-action space could be very large compared to the
experiment length and a sufficient sampling of the gain of each state-action
pair might become too costly to perform. Then it becomes increasingly important
how new or poorly sampled situations are approached.

The key observation is that Sarsa($\lambda$) is capable of quickly distributing
the value function over correlated states. However, it relies on that the policy
implements efficient heuristics for exploration. We tested a number of different
approaches to complement the standard Sarsa($\lambda$) algorithm, both when it
comes to exploration vs exploitation, and how to break ties. 

\subsection{Maximisers}
In Sarsa($\lambda$), whenever we reach a new state $S$, there is no clear
differentiation between different actions $a \in A$; statistically, all are
equally bad (if the outlook is pessimistic) or good (if we have an optimistic
outlook), but no matter how we choose to break the tie in the $Q(S,A)$ matrix
there will be some environments where we will benefit, and others where we will
lose.

The easiest approach is to just go with the first or last action a in
$\textit{arg max}_{a \in A} Q(S,a)$. Just from the example environments, there
are cases where this is a good idea, and cases where this is a bad idea. In
Tic-tac-toe, reverting to a specific order of tries, every time you end up in a
new state, is bad in the sense that you will play more illegal moves. But it can
also be good because due to how actions are enumerated, the first three of the
nine actions will combine into a three in a row. This assuming that the AI
doesn't play in that row before. Similarly, in the connect four environment,
playing in the same column over and over again results in a win quite often, but
if that column is filled, it will similarly cause a lot of illegal moves.

Another method we tried was to assume that in many environments, a state is
reminiscent of recently visited states. Even if we lack any information about
which action would be beneficial in a specific state, we might have better
sampling of the last states we visited, or the states before that. We define a
value of an untried action as
\begin{equation}
    v(a) = \sum_{i=1}^{t} \vert{A}\vert^{-i} n_{s_i}(a)
    \sqrt{n_{s_i}(a)}\Big/\sum_{i=1}^{t} \sqrt{n_{s_i}(a)}
\end{equation}

It assigns a higher value to actions that were deemed good in recent history,
giving more weight to well founded predictions.
