\paragraph{Complementing the core mechanism}
In some environments, the state-action space could be very large compared to the
experiment length and a sufficient mapping of the value of each state-action
pair might become too costly to perform. Then it becomes increasingly important
how new or poorly sampled situations are approached.

The key observation is that Sarsa($\lambda$) is capable of quickly distributing
the value function over correlated states. However, it relies on that the policy
implements efficient heuristics for exploration. We tested a number of different
approaches to complement the standard Sarsa($\lambda$) algorithm, both when it
comes to exploration vs exploitation, and how to break ties. 

\subsection{Maximisers}
In Sarsa($\lambda$), whenever we reach a new state $S$, there is no clear
differentiation between different actions $\{A\}$, statistically, all are
equally bad (if the outlook is pessimistic) or good (if we have an optimistic
outlook), but no matter how we choose to break the tie in the $Q(S,A)$ matrix,
there will be some environments where we will benefit, and some where we will
lose.

The easiest approach is to just go with the first or last action a $\textit{arg
    max}_{a\epsilon A} Q(S,a)$ value. Just from the example environments, there
are cases where this is a good idea, and cases where this is a bad idea. In tic
tac toe, reverting to a specific order of tries every time you end up in a new
state is bad in the sense that you will play more illegal moves, but it can also
be good because due to that how actions are enumerated, the first three test
actions will combine into a three in a row, if the AI doesn't happen to play in
that line before. Similarly, in the connect four environment, playing in the
same column over and over again results in a win quite often, but if that column
is filled, it will similarly cause a lot of illegal moves.

Another method we tried was to assume that in many environments, a state is
reminiscent of recently visited states. Even if we lack any information about
which action would be beneficial in a specific state, we might have better
sampling of the last states we visited, or the states before that. We define a value of an untried action as

\begin{equation}
	w(a) = \sum_{i \in |H|} nActions^{-i} Q(s_{h_i},a) \sqrt{C(s_{h_i},a)}/\sum_H \sqrt{C(H,a)}
\end{equation}

It places more worth on actions that were deemed good in recent history, giving
more weight to well founded predictions. We tried both to always choose the
highest worth action, and selecting an action with probability proportional to
(rescaled) worth. This method of breaking ties in the Q matrix proved useful in
board game environments  where the number of states were large, but for the
other environments, it didn't add any benefit and actually caused the agent to
perform worse in some examples. The softer version of choosing proportional to
gain were less successful in all cases. A problem with sticking to the highest
worth is that you tend to take one of few extreme paths: in the mines
environment you have a harder time learning to circumventing an obstacle, since
the algorithm causes you to go more in straight lines.
