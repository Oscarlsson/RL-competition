In addition to the implemented agent we have implemented two environments
representing the board games Tic-tac-toe and Connect Four. The Tic-tac-toe
environment is a fully observable Markov Decision Process (MDP) while Connect
Four is a Partially Observable MDP (POMDP), both are played against an
environment that always plays any available legal move. Illegal moves played by
the agent are punished by a negative reward, and the agent gets to try again
from the same board state.

\begin{tabular}{ l | l | l }
                & Tic-tac-toe & Connect Four \\
\hline
  State space 	& 19683 	& 2188 \\
  Action space 	& 9 		& 7 \\
  Agent wins 	& 10 		& 1 \\
  Illegal move 	& -10 	    & -1 \\
  AI wins	 	& -1 		& 0 \\
\end{tabular}

\paragraph{Tic-tac-toe}

The board state is considered as a ternary representation of an integer that
defines the state. Each digit describes whether a site at the board is empty
(0), occupied by player (1) or by the AI (2). The 9 different actions correspond
to playing at one of the 3x3 board game sites. The agent always gets the first
move, and wins by getting 3 in a row or when the board is filled.

\paragraph{Connect Four}

Connect Four is played on a vertical board game, game pieces are dropped down
from the top and fills up the board from the bottom. In this environment, the
agent are only allowed to observe the board state from the top, playing a piece
in one column will hide previous moves in that column from the agent. Similarly
to Tic-tac-toe, the agent receives an integer that represents a ternary digit
sequence describing the topmost board piece in each out of seven columns. The
starting player is randomised, and the game can end in a draw.
